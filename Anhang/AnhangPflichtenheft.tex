\subsection{Pflichtenheft (Auszug)}
\label{app:Pflichtenheft}

\subsubsection*{Zielbestimmung}

\begin{enumerate}[itemsep=0em,partopsep=0em,parsep=0em,topsep=0em]
    \item \textbf{Generierung und Verwaltung von QR-Codes}
    \begin{enumerate}
        \item Das Programm wird eine Funktion zur automatischen Generierung eines einzigartigen QR-Codes für jede abgeschlossene Bestellung implementieren.
        \item Die QR-Codes werden mithilfe einer Bibliothek namens \texttt{php-qrcode} erstellt. Diese Bibliothek ermöglicht es eine Zeichenkette in einen QR-Code zu verwandeln.
        \item Jeder generierte QR-Code wird in der Datenbank gespeichert. Dazu wird eine Tabelle \texttt{tl\_iso\_order\_hashes} verwendet, die die QR-Codes zusammen mit der zugehörigen Bestell-ID und weiteren Metadaten speichert.
        \item Die generierten QR-Codes werden automatisch per E-Mail an die Kunden versendet. Dafür wird eine E-Mail-Versandfunktion integriert, die die QR-Codes als Anhang oder Bild in die E-Mail einfügt.
    \end{enumerate}
    
    \item \textbf{Einlösung und Verifizierung von QR-Codes}
    \begin{enumerate}
        \item Das Programm wird eine Scanfunktion bereitstellen, die mithilfe der Bibliothek \texttt{html5-qrcode} die QR-Codes mit der Kamera eines mobilen Endgeräts scannt.
        \item Die Gültigkeit der gescannten QR-Codes wird überprüft, indem der QR-Code mit den in der Datenbank gespeicherten Hashes abgeglichen wird. Es wird sichergestellt, dass der QR-Code gültig und nicht bereits eingelöst ist.
        \item Nach dem Scannen eines QR-Codes zeigt das Programm die relevanten Bestellinformationen an, die aus der Datenbank abgerufen werden. Diese Informationen umfassen die Bestellnummer, den Namen des Kunden und die Details der Bestellung.
        \item Die Einlösung der QR-Codes wird im System markiert, indem der Status des QR-Codes in der Datenbank auf "eingelöst" gesetzt wird. Dies verhindert die erneute Nutzung des gleichen QR-Codes.
        \item Das Programm stellt sicher, dass ungültige oder bereits eingelöste QR-Codes nicht akzeptiert werden, indem es eine entsprechende Fehlermeldung anzeigt.
    \end{enumerate}

    \item \textbf{Benutzeroberfläche und Usability}
    \begin{enumerate}
        \item Die Benutzeroberfläche wird mithilfe von HTML5-Templates und SCSS entwickelt. Diese Templates werden so gestaltet, dass sie intuitiv und benutzerfreundlich sind.
        \item Die Navigation innerhalb des Programms wird klar und logisch strukturiert, sodass Benutzer leicht zwischen den verschiedenen Funktionen wechseln können.
        \item Fehlermeldungen werden klar formuliert und hilfreich gestaltet, um den Benutzern die Lösung von Problemen zu erleichtern. Für das anzeigen dynamischer Fehlermeldungen wird JavaScript verwendet.
    \end{enumerate}

    \item \textbf{Sicherheit und Zugriffskontrolle}
    \begin{enumerate}
        \item Das Programm wird Maßnahmen zum Schutz vor gefälschten und manipulierten QR-Codes integrieren.
        \item Es wird sichergestellt, dass nur autorisierte Mitarbeiter Zugang zum Programm haben. Dies wird durch die Mitgliedsverwaltung in Contao realisiert, die den Zugriff auf bestimmte Funktionen und Daten einschränkt.
        \item Die Zugriffskontrolle verhindert, dass unbefugte Personen sensible Bestellinformationen einsehen oder Tickets versehentlich entwerten können.
    \end{enumerate}

    \item \textbf{Integrationsfähigkeit}
    \begin{enumerate}
        \item Das Programm wird als Symfony Bundle entwickelt, das sich nahtlos in die bestehende Contao-Umgebung integrieren lässt. Dies erleichtert die Installation und Wartung.
        \item Die Struktur des Programms entspricht den Anforderungen eines Symfony Bundles, sodass es problemlos in andere Projekte eingebunden werden kann.
    \end{enumerate}

    \item \textbf{Zuverlässigkeit und Effizienz}
    \begin{enumerate}
        \item Das Programm wird so entwickelt, dass es zuverlässig und korrekt arbeitet. Dies wird durch manuelle Frontend-Tests sichergestellt.
    \end{enumerate}
\end{enumerate}

\subsubsection*{Produkteinsatz}

\begin{enumerate}[itemsep=0em,partopsep=0em,parsep=0em,topsep=0em]
    \item Anwendungsbereiche\\
    Das QR-Code-basierte Einlösungsprogramm dient zur Verwaltung und Verifizierung von Online-Tickets. Es soll eine effiziente und sichere Abwicklung der Ticketverkäufe und deren Einlösung vor Ort ermöglichen.

    \item Zielgruppen\\
    Die Hauptnutzer der Anwendung sind unsere Kunden, die für die Verifizierung und Einlösung der Tickets ihrer Online-Shops verantwortlich sind. Dazu gehören insbesondere Mitarbeiter in der Veranstaltungsbranche, im Einzelhandel sowie im Tourismusbereich.

    \item Betriebsbedingungen\\
    Die Anwendung muss auf Smartphones mit Webbrowser und Internetverbindung lauffähig sein. Das Programm wird als Webanwendung bereitgestellt. Eine ständige Verfügbarkeit der Anwendung ist durch den Betrieb auf einem Webserver sichergestellt. Updates und Wartungen werden zentral durchgeführt, um die kontinuierliche Verfügbarkeit zu gewährleisten.
\end{enumerate}