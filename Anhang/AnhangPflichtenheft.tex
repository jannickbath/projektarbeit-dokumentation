\subsection{Pflichtenheft (Auszug)}
\label{app:Pflichtenheft}

\subsubsection*{Zielbestimmung}

\begin{enumerate}[itemsep=0em,partopsep=0em,parsep=0em,topsep=0em]
\item Musskriterien
    \begin{enumerate}
        \item Generierung und Verwaltung von QR-Codes
        \begin{itemize}
            \item Das Programm muss für jede abgeschlossene Bestellung automatisch einen einzigartigen QR-Code erstellen.
            \item Jeder generierte QR-Code muss in der Datenbank erfasst und gespeichert werden.
            \item Die QR-Codes müssen per E-Mail an die Kunden verschickt werden.
        \end{itemize}
        
        \item Einlösung und Verifizierung von QR-Codes
        \begin{itemize}
            \item Das Programm muss das Scannen der QR-Codes mittels der Kamera eines mobilen Endgeräts ermöglichen.
            \item Das Programm muss die Gültigkeit der gescannten QR-Codes überprüfen.
            \item Beim Scannen eines QR-Codes müssen die relevanten Bestellinformationen angezeigt werden.
            \item Das Programm muss die Einlösung der QR-Codes bestätigen und diese als eingelöst markieren.
            \item Ungültige oder bereits eingelöste QR-Codes dürfen nicht erneut akzeptiert werden.
        \end{itemize}

        \item Benutzeroberfläche und Usability
        \begin{itemize}
            \item Die Benutzeroberfläche des Programms muss intuitiv und benutzerfreundlich gestaltet sein.
            \item Die Navigation innerhalb des Programms muss klar und strukturiert sein.
            \item Fehlermeldungen müssen verständlich und hilfreich formuliert sein.
        \end{itemize}

        \item Sicherheit und Zugriffskontrolle
        \begin{itemize}
            \item Das Programm muss vor gefälschten und manipulierten QR-Codes sicher sein.
            \item Es muss sichergestellt werden, dass nur autorisierte Mitarbeiter Zugang zum Programm haben.
            \item Die Zugriffskontrolle muss verhindern, dass unbefugte Personen sensible Bestellinformationen einsehen oder Tickets versehentlich entwerten können.
        \end{itemize}

        \item Integrationsfähigkeit
        \begin{itemize}
            \item Das Programm muss sich nahtlos in die bestehende Systemlandschaft integrieren.
            \item Die Struktur des Programms muss den Anforderungen eines Symfony Bundles entsprechen.
        \end{itemize}

        \item Zuverlässigkeit und Effizienz
        \begin{itemize}
            \item Das Programm muss zuverlässig und korrekt arbeiten.
            \item Kritische Fehler, die das gesamte Programm lahmlegen, müssen unbedingt vermieden werden.
        \end{itemize}
    \end{enumerate}
\end{enumerate}

\subsubsection*{Produkteinsatz}

\begin{enumerate}[itemsep=0em,partopsep=0em,parsep=0em,topsep=0em]
    \item Anwendungsbereiche\\
    Das QR-Code-basierte Einlösungsprogramm dient zur Verwaltung und Verifizierung von Online-Tickets. Es soll eine effiziente und sichere Abwicklung der Ticketverkäufe und deren Einlösung vor Ort ermöglichen.

    \item Zielgruppen\\
    Die Hauptnutzer der Anwendung sind unsere Kunden, die für die Verifizierung und Einlösung der Tickets ihrer Online-Shops verantwortlich sind. Dazu gehören insbesondere Mitarbeiter in der Veranstaltungsbranche, im Einzelhandel sowie im Tourismusbereich.

    \item Betriebsbedingungen\\
    Die Anwendung muss auf Smartphones mit Webbrowser und Internetverbindung lauffähig sein. Das Programm wird als Webanwendung bereitgestellt. Eine ständige Verfügbarkeit der Anwendung ist durch den Betrieb auf einem Webserver sichergestellt. Updates und Wartungen werden zentral durchgeführt, um die kontinuierliche Verfügbarkeit zu gewährleisten.
\end{enumerate}