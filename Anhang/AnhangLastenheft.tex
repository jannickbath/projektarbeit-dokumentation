\subsection{Lastenheft (Auszug)}
\label{app:Lastenheft}

Die Anwendung muss folgende Anforderungen erfüllen:
\begin{enumerate}[itemsep=0em,partopsep=0em,parsep=0em,topsep=0em]
    \item Generierung und Verwaltung von QR-Codes
    \begin{enumerate}
        \item Die Anwendung muss für jede abgeschlossene Bestellung einen einzigartigen QR-Code generieren.
        \item Es muss ein Eintrag für jedes Ticket in die Datenbank geschrieben werden.
        \item Die QR-Codes müssen per E-Mail an die Kunden versendet werden.
    \end{enumerate}

    \item Einlösung und Verifizierung von QR-Codes
    \begin{enumerate}
        \item Die Anwendung muss das Scannen der QR-Codes ermöglichen.
        \item Die Anwendung muss die Gültigkeit der gescannten QR-Codes überprüfen.
        \item Die Anwendung muss die Bestellinformationen anzeigen, wenn ein QR-Code gescannt wird.
        \item Die Anwendung muss es ermöglichen, die Einlösung der QR-Codes zu bestätigen und diese als eingelöst zu markieren.
        \item Ungültige oder bereits eingelöste QR-Codes dürfen nicht erneut akzeptiert werden.
    \end{enumerate}

    \item Benutzeroberfläche und Usability
    \begin{enumerate}
        \item Die Benutzeroberfläche muss intuitiv und benutzerfreundlich gestaltet sein.
        \item Die Navigation durch die Anwendung muss einfach und klar strukturiert sein.
        \item Fehlermeldungen müssen verständlich und hilfreich sein, um dem Benutzer bei der Problembehebung zu helfen.
    \end{enumerate}

    \item Sicherheit und Zugriffskontrolle
    \begin{enumerate}
        \item Die Anwendung muss vor gefälschten und manipulierten QR-Codes sicher sein.
        \item Es muss sichergestellt werden, dass nur autorisierte Mitarbeiter mit einem hinterlegten Login Zugang zur Anwendung haben.
        \item Die Zugriffskontrolle muss verhindern, dass unbefugte Personen sensible Bestellinformationen einsehen oder Tickets versehentlich entwerten können.
    \end{enumerate}

    \item Integrationsfähigkeit
    \begin{enumerate}
        \item Die Anwendung muss sich nahtlos in die bestehende Systemlandschaft integrieren.
        \item Die Struktur der Anwendung muss den Anforderungen eines Symfony Bundles entsprechen.
    \end{enumerate}

    \item Zuverlässigkeit und Effizienz
    \begin{enumerate}
        \item Die Anwendung muss verlässlich und korrekt arbeiten.
        \item Kritische Fehler, die das gesamte Programm lahmlegen, müssen unbedingt vermieden werden.
    \end{enumerate}
\end{enumerate}

