\subsection{Benutzerdokumentation}
\label{app:BenutzerDoku}

\justifying

\subsubsection{Starten des Programms}
Öffnen Sie den Webbrowser Ihrer Wahl und navigieren Sie zu \url{https://[domain]/einloesen}. Wenn Sie nicht eingeloggt sind, dann werden Sie nun zur Login-Seite weitergeleitet. Melden Sie sich hier mit Ihrem Mitgliedszugang von Contao an.

Nachdem Sie sich erfolgreich eingeloggt haben, werden Sie zur Startseite weitergeleitet.

Stellen Sie sicher, dass Ihre Kamera und Internetverbindung aktiviert sind, um alle Funktionen der Anwendung nutzen zu können.

\subsubsection{Scannen eines QR-Codes}
\textbf{QR-Code-Scanner öffnen:} Klicken sie auf der Startseite auf das QR-Code-Icon um den Scan-Prozess zu starten.

\textbf{Kamera ausrichten:} Richten Sie die Kamera Ihres Geräts auf den QR-Code des Tickets. Achten Sie darauf, dass der QR-Code vollständig im Sichtfeld der Kamera ist.

\textbf{QR-Code scannen:} Das Programm scannt den QR-Code automatisch und leitet Sie zur Ticketinformation weiter. Hier sehen Sie alle relevanten Details zur Bestellung.


\subsubsection{Entwertung des Tickets}

Nachdem der QR-Code erfolgreich gescannt wurde, können Sie das Ticket entwerten:

\textbf{Ticketinformationen überprüfen:} Stellen Sie sicher, dass die angezeigten Informationen mit den Daten des Gastes übereinstimmen.

\textbf{Ticket entwerten:} Klicken Sie auf die Schaltfläche "Ticket entwerten", um den QR-Code im System als eingelöst zu markieren.


\subsubsection{Fehlerbehebung beim Scannen}

Sollte der Scan fehlschlagen, gibt es einige Ansätze, um das Problem zu beheben:

\textbf{Lichtverhältnisse anpassen:} Starkes Sonnenlicht kann den Bildschirm eines Handys schwer erkennbar machen. Stellen Sie sich, wenn nötig, in den Schatten, um den QR-Code besser sichtbar zu machen.

\textbf{QR-Code vergrößern:} Wenn der QR-Code zu klein dargestellt wird, können Sie den Bildschirm vergrößern, um den QR-Code besser erkennbar zu machen. Nutzen Sie die Zoom-Funktion Ihres Geräts.


\subsubsection{Manuelle Suche für Tickets}

Falls der Scan-Prozess gerade nicht funktioniert, haben sie die Möglichkeit manuell nach Tickets zu suchen. Hierbei kann nach Datum, Bestellnummer oder Name gesucht werden.

\textbf{Zur Ticketliste wechseln:} Klicken Sie auf der Hauptseite oder im QR-Code-Scanner auf den Button mit der Lupe, um zur Ticketliste zu gelangen.

\textbf{Ticketnummer eingeben:} Geben Sie die Ticketnummer, Datum oder den Namen des Gastes in die Suchleiste ein.

\textbf{Ticket entwerten:} Wählen Sie das entsprechende Ticket aus der Liste aus und klicken Sie auf "Ticket entwerten", um es manuell zu entwerten.

