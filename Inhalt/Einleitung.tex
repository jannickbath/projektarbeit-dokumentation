% !TEX root = ../Projektdokumentation.tex
\section{Einleitung}
\label{sec:Einleitung}


\subsection{Projektumfeld} 
\label{sec:Projektumfeld}

\subsubsection{Ausbildungsbetrieb}

Die \textbf{LUPCOM media GmbH} ist eine Internetagentur mit ihrem Hauptsitz in Rostock, die sich darauf spezialisiert hat, Webseiten und Webanwendungen für eine vielfältige Kundschaft zu entwerfen und zu entwickeln. Ihr Fokus liegt darauf, maßgeschneiderte Lösungen zu bieten, die den individuellen Anforderungen und Größenordnungen ihrer Kunden entsprechen. Dabei deckt sie ein breites Spektrum von Projekten ab, von kleinen Unternehmenswebseiten bis hin zu komplexen Webanwendungen. Für die Umsetzung dieser Projekte werden Technologien wie Contao, Symfony, JavaScript sowie Docker verwendet.

\subsubsection{Kunde}

Der Auftraggeber dieses Projekts ist die \textbf{MS "Ostseebad Rerik"}, ein Ausflugsschiff, das entlang der „verbotenen“ Halbinsel Wustrow fährt. Das Schiff bietet mehrmals täglich in der Hauptsaison Rundfahrten auf dem Salzhaff an, die etwa zwei Stunden dauern und unter kundiger Führung des Kapitäns stattfinden. Zusätzlich bietet die MS "Ostseebad Rerik" gastronomische Versorgung sowie Charterfahrten für Jubiläen, Betriebsfeiern und Vereinsausflüge an.


\subsection{Projektziel} 
\label{sec:Projektziel}

Ziel dieses Projekts ist die Entwicklung eines zuverlässigen und effizienten QR-Code-basierten Einlösungsprogramms. Dieses Programm soll den Einlösungsprozess digitalisieren und automatisieren. Konkret umfasst das Projektziel die folgenden Punkte:

\begin{itemize}
    \item \textbf{Generierung einzigartiger QR-Codes für spezifische Bestellungen:} Für jede Bestellung soll ein individueller QR-Code generiert werden, der durch einen Mitarbeiter eingescannt und entwertet werden kann.
    \item \textbf{Sichere und effiziente Einlösung durch Scannen der QR-Codes:} Die QR-Codes sollen von den Mitarbeitern gescannt werden können, um die Echtheit und Gültigkeit der Tickets oder Gutscheine sofort zu überprüfen.
    \item \textbf{Intuitive Benutzeroberfläche für die Überprüfung der Bestellinformationen und die Entwertung der Tickets/Gutscheine:} Das System soll eine benutzerfreundliche Oberfläche bieten, die es den Mitarbeitern ermöglicht, die Bestellinformationen schnell und einfach einzusehen und die Tickets oder Gutscheine nach der Einlösung zu entwerten.
\end{itemize}

Durch die Erfüllung dieser Ziele wird das Projekt erfolgreich abgeschlossen.

\subsection{Projektbegründung} 
\label{sec:Projektbegruendung}
\begin{itemize}
	\item Warum ist das Projekt sinnvoll (\zB Kosten- oder Zeitersparnis, weniger Fehler)?
	\item Was ist die Motivation hinter dem Projekt?
\end{itemize}


\subsection{Projektschnittstellen} 
\label{sec:Projektschnittstellen}
\begin{itemize}
	\item Mit welchen anderen Systemen interagiert die Anwendung (technische Schnittstellen)?
	\item Wer genehmigt das Projekt \bzw stellt Mittel zur Verfügung? 
	\item Wer sind die Benutzer der Anwendung?
	\item Wem muss das Ergebnis präsentiert werden?
\end{itemize}


\subsection{Projektabgrenzung} 
\label{sec:Projektabgrenzung}
\begin{itemize}
	\item Was ist explizit nicht Teil des Projekts (\insb bei Teilprojekten)?
\end{itemize}
