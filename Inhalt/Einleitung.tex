% !TEX root = ../Projektdokumentation.tex
\section{Einleitung}
\label{sec:Einleitung}


\subsection{Projektumfeld} 
\label{sec:Projektumfeld}

\subsubsection{Ausbildungsbetrieb}

Die \textbf{LUPCOM media GmbH} ist eine Internetagentur mit ihrem Hauptsitz in Rostock, die sich darauf spezialisiert hat, Webseiten und Webanwendungen für eine vielfältige Kundschaft zu entwerfen und zu entwickeln. Ihr Fokus liegt darauf, maßgeschneiderte Lösungen zu bieten, die den individuellen Anforderungen und Größenordnungen ihrer Kunden entsprechen. Dabei deckt sie ein breites Spektrum von Projekten ab, von kleinen Unternehmenswebseiten bis hin zu komplexen Webanwendungen. Für die Umsetzung dieser Projekte werden Technologien wie Contao, Symfony, JavaScript sowie Docker verwendet.

\subsubsection{Kunde}

Der Auftraggeber dieses Projekts ist die \textbf{MS "Ostseebad Rerik"}, ein Ausflugsschiff, das entlang der „verbotenen“ Halbinsel Wustrow fährt. Das Schiff bietet mehrmals täglich in der Hauptsaison Rundfahrten auf dem Salzhaff an, die etwa zwei Stunden dauern und unter kundiger Führung des Kapitäns stattfinden. Zusätzlich bietet die MS "Ostseebad Rerik" gastronomische Versorgung sowie Charterfahrten für Jubiläen, Betriebsfeiern und Vereinsausflüge an.


\subsection{Projektziel} 
\label{sec:Projektziel}

Ziel dieses Projekts ist die Entwicklung eines zuverlässigen und effizienten QR-Code-basierten Einlösungsprogramms. Dieses Programm soll den Einlösungsprozess digitalisieren und automatisieren. Konkret umfasst das Projektziel die folgenden Punkte:

\begin{itemize}
    \item \textbf{Generierung einzigartiger QR-Codes für spezifische Bestellungen:} Für jede Bestellung soll ein individueller QR-Code generiert werden, der durch einen Mitarbeiter eingescannt und entwertet werden kann.
    \item \textbf{Sichere und effiziente Einlösung durch Scannen der QR-Codes:} Die QR-Codes sollen von den Mitarbeitern gescannt werden können, um die Echtheit und Gültigkeit der Tickets oder Gutscheine sofort zu überprüfen.
    \item \textbf{Intuitive Benutzeroberfläche für die Überprüfung der Bestellinformationen und die Entwertung der Tickets/Gutscheine:} Das System soll eine benutzerfreundliche Oberfläche bieten, die es den Mitarbeitern ermöglicht, die Bestellinformationen schnell und einfach einzusehen und die Tickets oder Gutscheine nach der Einlösung zu entwerten.
	\item \textbf{Einfache Installation mittels composer:} Die Anwendung muss mit dem Paketmanager composer kompatibel sein, um eine einfache Installation zu gewährleisten.
\end{itemize}

Durch die Erfüllung dieser Ziele wird das Projekt erfolgreich abgeschlossen.

\subsection{Projektbegründung} 
\label{sec:Projektbegruendung}

Die Durchführung dieses Projekts soll mehrere bestehende Probleme im aktuellen Einlösungsprozess beheben und sowohl die Effizienz als auch die Zufriedenheit der Mitarbeiter steigern.

\subsection{Aktuelles Problem}
Der aktuelle Prozess zur Überprüfung und Einlösung von online erworbenen Tickets und Gutscheinen basiert auf manuellen Methoden. Diese manuelle Verifizierung ist zeitaufwendig und fehleranfällig, was zu einer ineffizienten Nutzung der Arbeitszeit führt und die Kundenzufriedenheit negativ beeinflusst.

\subsection{Vorteile des QR-Code-basierten Systems}
Die Entwicklung eines QR-Code-basierten Einlösungsprogramms bietet zahlreiche Vorteile:

\begin{itemize}
    \item \textbf{Reduktion von Fehlern:} Durch die Automatisierung des Verifizierungsprozesses werden menschliche Fehler minimiert, was die Zuverlässigkeit der Ticket- und Gutscheinüberprüfung erhöht.
    \item \textbf{Zeitersparnis:} Mitarbeiter können QR-Codes schnell scannen und sofortige Rückmeldungen über die Gültigkeit der Tickets erhalten, was den gesamten Prozess beschleunigt.
    \item \textbf{Erhöhte Sicherheit:} Das System gewährleistet eine sichere Einlösung von Tickets und Gutscheinen, wodurch Betrugsversuche reduziert werden.
    \item \textbf{Verbesserte Kundenzufriedenheit:} Ein effizienter und schneller Einlösungsprozess steigert das Kundenerlebnis und das Vertrauen in die digitalen Dienstleistungen.
\end{itemize}

\subsection{Vorteile für die Mitarbeiter}
Die Einführung eines QR-Code-basierten Einlösungsprogramms soll die Mitarbeiter vor Ort entlasten:

\begin{itemize}
    \item \textbf{Erleichterung der Arbeitsabläufe:} Die Automatisierung reduziert den manuellen Aufwand und die Komplexität der Aufgaben, was die Arbeitslast der Mitarbeiter verringert.
    \item \textbf{Steigerung der Produktivität:} Mitarbeiter können ihre Zeit effizienter nutzen und sich auf andere wichtige Aufgaben konzentrieren.
    \item \textbf{Verbesserte Arbeitsqualität:} Durch die Minimierung von Fehlern und die Vereinfachung der Prozesse wird die Qualität der Arbeit verbessert, was zu höherer Arbeitszufriedenheit führt.
\end{itemize}

Insgesamt soll die Implementierung eines QR-Code-basierten Einlösungsprogramms zu einer Verbesserung der Effizienz und Qualität der Arbeitsabläufe sowie zu einer gesteigerten Zufriedenheit der Mitarbeiter und Kunden führen.

\subsection{Projektschnittstellen} 
\label{sec:Projektschnittstellen}

Die Anwendung soll verschiedene Schnittstellen bereitstellen, um eine nahtlose Integration und effiziente Nutzung zu gewährleisten. Im Folgenden werden die geplanten Schnittstellen beschrieben.

\subsubsection{Technische Schnittstellen}

\begin{itemize}
    \item \textbf{Interaktion mit Shop-Systemen:} Die Anwendung soll sich nahtlos in die bestehenden Shop-Systeme integrieren. Hierbei soll für jede Bestellung ein einzigartiger QR-Code generiert werden, der in den Bestelldaten gespeichert und den Kunden zur Verfügung gestellt wird.
    \item \textbf{API-Schnittstellen:} Die Anwendung soll mehrere API-Endpunkte bereitstellen, um die Interaktion mit externen Systemen und die Verarbeitung der QR-Codes zu ermöglichen. Dies umfasst Endpunkte für die Validierung und Entwertung der QR-Codes sowie die Bereitstellung der entsprechenden Bestelldetails.
\end{itemize}

\subsubsection{Präsentation der Ergebnisse}

\begin{itemize}
    \item \textbf{Benutzeroberfläche:} Die Ergebnisse der QR-Code-Verifizierung und Einlösung sollen über eine benutzerfreundliche Oberfläche präsentiert werden.
    \item \textbf{Darstellung für Mitarbeiter:} Die Bestellinformationen sollen den Mitarbeitern in einem klaren und strukturierten Format präsentiert werden, um eine schnelle und effiziente Überprüfung und Entwertung der Tickets zu ermöglichen. Dies umfasst die Anzeige der Bestelldetails wie Bestellnummer, Datum und Status sowie den Entwertungsstatus.
\end{itemize}


\subsection{Projektabgrenzung} 
\label{sec:Projektabgrenzung}
\begin{itemize}
	\item Was ist explizit nicht Teil des Projekts (\insb bei Teilprojekten)?
\end{itemize}
