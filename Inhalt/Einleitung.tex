% !TEX root = ../Projektdokumentation.tex
\section{Einleitung}
\label{sec:Einleitung}


\subsection{Projektumfeld} 
\label{sec:Projektumfeld}

\subsubsection{Ausbildungsbetrieb}

Die \textbf{LUPCOM media GmbH} ist eine Internetagentur mit ihrem Hauptsitz in Rostock, die sich darauf spezialisiert hat, Webseiten und Webanwendungen für eine vielfältige Kundschaft zu entwerfen und zu entwickeln. Ihr Fokus liegt darauf, maßgeschneiderte Lösungen zu bieten, die den individuellen Anforderungen und Größenordnungen ihrer Kunden entsprechen. Dabei deckt sie ein breites Spektrum von Projekten ab, von kleinen Unternehmenswebseiten bis hin zu komplexen Webanwendungen. Für die Umsetzung dieser Projekte werden Technologien wie Contao, Symfony, JavaScript sowie Docker verwendet.

\subsubsection{Kunde}

Der Auftraggeber dieses Projekts ist die \textbf{MS `Ostseebad Rerik`}, ein Ausflugsschiff, das entlang der „verbotenen“ Halbinsel Wustrow fährt. Das Schiff bietet mehrmals täglich in der Hauptsaison Rundfahrten auf dem Salzhaff an, die etwa zwei Stunden dauern und unter kundiger Führung des Kapitäns stattfinden. Zusätzlich bietet die MS `Ostseebad Rerik` gastronomische Versorgung sowie Charterfahrten für Jubiläen, Betriebsfeiern und Vereinsausflüge an.


\subsection{Projektziel} 
\label{sec:Projektziel}

Das Ziel des Projekts ist es, den Einlösungsprozess von Tickets zu digitalisieren und automatisieren. Durch das Scannen der QR-Codes sollen die Mitarbeiter die Echtheit und Gültigkeit der Tickets sofort überprüfen können. Das System soll eine benutzerfreundliche Oberfläche bieten, die es den Mitarbeitern ermöglicht, Bestellinformationen schnell und einfach einzusehen und die Tickets zu entwerten.

\subsection{Projektbegründung} 
\label{sec:Projektbegruendung}

Die Durchführung dieses Projekts soll mehrere bestehende Probleme im aktuellen Einlösungsprozess beheben und sowohl die Effizienz als auch die Zufriedenheit der Mitarbeiter steigern.

Der aktuelle Prozess zur Überprüfung und Einlösung von Tickets ist manuell, zeitaufwendig und fehleranfällig. Dies führt zu einer ineffizienten Nutzung der Arbeitszeit und beeinträchtigt die Kundenzufriedenheit.

Durch die Automatisierung des Verifizierungsprozesses können menschliche Fehler minimiert werden, was die Zuverlässigkeit der Ticketüberprüfung erhöht. Mitarbeiter können QR-Codes schnell scannen und sofort Rückmeldung über die Gültigkeit der Tickets erhalten, was den gesamten Prozess beschleunigt. Das System soll eine sichere Einlösung von Tickets gewährleisten, wodurch Betrugsversuche reduziert werden können.

\subsection{Projektschnittstellen} 
\label{sec:Projektschnittstellen}

Die Anwendung wird verschiedene Schnittstellen bereitstellen, um eine nahtlose Integration und effiziente Nutzung zu gewährleisten. Die Anwendung soll sich nahtlos in das bestehende Shop-System Isotope und in das Framework Contao integrieren.

Für jede Bestellung soll ein einzigartiger QR-Code generiert werden, der in der Datenbank gespeichert und den Kunden zur Verfügung gestellt wird. Darüber hinaus soll die Anwendung mehrere API-Endpunkte bereitstellen, um die Interaktion mit externen Systemen und die Verarbeitung der QR-Codes zu ermöglichen. Dies umfasst Endpunkte für die Validierung und Entwertung der QR-Codes sowie die Bereitstellung der entsprechenden Bestelldetails.

Die Ergebnisse der Verifizierung und Einlösung sollen über eine benutzerfreundliche Oberfläche präsentiert werden. Die Bestellinformationen sollen den Mitarbeitern in einem klaren und strukturierten Format präsentiert werden, um eine schnelle und effiziente Überprüfung und Entwertung der Tickets zu ermöglichen. Dies umfasst die Anzeige der Bestelldetails wie Bestellnummer, Datum und Status sowie den Entwertungsstatus.

\subsection{Projektabgrenzung} 
\label{sec:Projektabgrenzung}

Die Entwicklung und Konfiguration des Onlineshops sind nicht Bestandteil dieses Projekts. Alle Arbeiten, die mit der Einrichtung, Verwaltung und Anpassung des Onlineshops selbst zu tun haben, fallen somit außerhalb des Projektumfangs.

Ebenso gehört die Erstellung der Designs für das Einlösetool nicht zum Projektumfang. Die grafischen Entwürfe und Layouts werden durch die Design-Abteilung bereitgestellt.