% !TEX root = ../Projektdokumentation.tex
\section{Projektplanung} 
\label{sec:Projektplanung}


\subsection{Projektphasen}
\label{sec:Projektphasen}

Für die Umsetzung des Projekts standen insgesamt 80 Stunden zur Verfügung, die vor Beginn auf verschiedene Phasen der Softwareentwicklung verteilt wurden. Die Projektarbeit erstreckte sich über zwei 4-Tage-Wochen, ergänzt durch zwei zusätzliche Arbeitstage, um Feiertage auszugleichen. Die Durchführung erfolgte vom 25.04.2024 bis 26.04.2024, sowie vom 29.04.2024 bis 03.05.2024 und vom 20.05.2024 bis 24.05.2024, jeweils mit einer täglichen Arbeitszeit von 8 Stunden.

Eine Übersicht der groben Zeitplanung und der Hauptphasen findet sich in Tabelle 1: Grobe Zeitplanung. Diese Hauptphasen sind weiter in detaillierte Unterabschnitte gegliedert, um eine präzise Planung zu gewährleisten. Eine detaillierte Darstellung der Phasen ist im Anhang A.1: Detaillierte Zeitplanung auf Seite i zu finden.

\paragraph{Tabelle~\ref{tab:Zeitplanung}} zeigt die grobe Zeitplanung.
\tabelle{Zeitplanung}{tab:Zeitplanung}{ZeitplanungKurz}\\
Eine detailliertere Zeitplanung findet sich im \Anhang{app:Zeitplanung}.


\subsection{Abweichungen vom Projektantrag}
\label{sec:AbweichungenProjektantrag}

Zunächst war in der Zeitplanung des Projektantrags die Erstellung eines Abnahmeprotokolls vorgesehen. Dieses Protokoll wurde jedoch nicht erstellt. Stattdessen erfolgte die Übergabe der Projektergebnisse direkt an die zuständige Abteilung, die die Abnahme informell durch ein einfaches Abnicken bestätigte.

Des Weiteren waren im Projektantrag Unit-Tests eingeplant, um die Qualität des Codes sicherzustellen. Aufgrund von Zeitmangel konnten diese Tests jedoch nicht durchgeführt werden. Stattdessen wurden manuelle Tests durchgeführt, bei denen die Arbeitsabläufe durch manuelles Durchklicken überprüft wurden.

Eine weitere Abweichung betraf die Erstellung der Entwürfe. Im Projektantrag war vorgesehen, dass diese aufgrund ihrer erwarteten geringen Anzahl vom Autor erstellt werden. Allerdings stellte sich während des Projekts heraus, dass die Anzahl der erforderlichen Benutzeroberflächen umfangreicher war als angenommen. Daher wurde beschlossen, die Entwürfe an die Design-Abteilung zu übergeben. Lediglich die Ticketansicht und die Popups wurden vom Autor entworfen.

Auch die Benutzerdokumentation fiel kleiner aus als ursprünglich geplant. Dies lag daran, dass die Abnahme durch die Fachabteilung mehr Zeit in Anspruch nahm als vorgesehen, was die Zeit für die Erstellung der Dokumentation verkürzte.

\subsection{Ressourcenplanung}
\label{sec:Ressourcenplanung}

Die Entwicklung des Projekts wurde auf einem Desktop-Rechner mit dem Betriebssystem Ubuntu durchgeführt. Visual Studio Code (VSCode) diente als Hauptwerkzeug für die Code-Entwicklung, während Git für die Versionskontrolle und GitLab zur Sicherung und Verwaltung des Projektstands verwendet wurden.

Das Open Source Contao CMS wurde als Basis für die Entwicklung der Anwendung verwendet. Zur Integration in die Online-Shops diente das Contao-Plugin Isotope. Die Anwendung wurde in verschiedenen Browsern wie Chrome, Safari und Firefox getestet, wobei auch die Google Page-Speed Tools zur Optimierung der Leistungsfähigkeit und Ladezeiten zum Einsatz kamen.

Die personellen Ressourcen setzten sich aus dem Prüfling als Entwickler, dem Geschäftsführer als Projektverantwortlichem und einem Designer für die grafischen Entwürfe zusammen.
\paragraph{\Anhang{app:Ressourcenplanung}}

\subsection{Entwicklungsprozess}
\label{sec:Entwicklungsprozess}

Vor der Durchführung des Projekts wurde ein spezifisches Vorgehensmodell ausgewählt, das die Struktur und den Ablauf der Entwicklung vorgab. Für dieses Projekt entschied sich der Autor für das Wasserfallmodell, da es in Bezug auf die Komplexität und den Umfang des Projekts als am besten geeignet erschien.

Das Wasserfallmodell zeichnet sich durch seine lineare Herangehensweise aus, bei der jede Phase vollständig abgeschlossen sein muss, bevor die nächste beginnt. Dies ermöglichte eine klare und strukturierte Planung sowie eine präzise Definition der einzelnen Projektphasen.

Ein weiterer Punkt, der zu dieser Entscheidung führte, war die Tatsache, dass die Entwicklung hauptsächlich von einer Person, dem Prüfling, durchgeführt wurde. Dadurch konnte das Wasserfallmodell seine Stärken ausspielen, da es eine klare und übersichtliche Projektstruktur bietet, die besonders bei Einzelentwicklungen von Vorteil ist.

Die einzelnen Phasen des Wasserfallmodells, wie Anforderungsanalyse, Entwurf, Implementierung, Testen und Wartung, wurden nacheinander durchlaufen, um eine systematische und gründliche Entwicklung zu gewährleisten. Diese Methodik erlaubte es, den Projektverlauf genau zu planen und mögliche Risiken frühzeitig zu identifizieren und zu minimieren.
