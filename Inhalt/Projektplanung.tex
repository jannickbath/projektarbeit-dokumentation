% !TEX root = ../Projektdokumentation.tex
\section{Projektplanung} 
\label{sec:Projektplanung}


\subsection{Projektphasen}
\label{sec:Projektphasen}

Für die Umsetzung des Projekts standen insgesamt 80 Stunden zur Verfügung, die vor Beginn auf verschiedene Phasen der Softwareentwicklung verteilt wurden. Die Projektarbeit erstreckte sich über zwei 4-Tage-Wochen, ergänzt durch zwei zusätzliche Arbeitstage, um Feiertage auszugleichen. Die Durchführung erfolgte vom 25.04.2024 bis 26.04.2024, sowie vom 29.04.2024 bis 03.05.2024 und vom 20.05.2024 bis 24.05.2024, jeweils mit einer täglichen Arbeitszeit von 8 Stunden.

Eine Übersicht der groben Zeitplanung und der Hauptphasen findet sich in Tabelle 1: Grobe Zeitplanung. Diese Hauptphasen sind weiter in detaillierte Unterabschnitte gegliedert, um eine präzise Planung zu gewährleisten. Eine detaillierte Darstellung der Phasen ist im Anhang A.1: Detaillierte Zeitplanung auf Seite i zu finden.

\paragraph{Tabelle~\ref{tab:Zeitplanung}} zeigt die grobe Zeitplanung.
\tabelle{Zeitplanung}{tab:Zeitplanung}{ZeitplanungKurz}\\
Eine detailliertere Zeitplanung findet sich im \Anhang{app:Zeitplanung}.


\subsection{Abweichungen vom Projektantrag}
\label{sec:AbweichungenProjektantrag}

\begin{itemize}
	\item Sollte es Abweichungen zum Projektantrag geben (\zB Zeitplanung, Inhalt des Projekts, neue Anforderungen), müssen diese explizit aufgeführt und begründet werden.
\end{itemize}


\subsection{Ressourcenplanung}
\label{sec:Ressourcenplanung}

\begin{itemize}
	\item Detaillierte Planung der benötigten Ressourcen (Hard-/Software, Räumlichkeiten \usw).
	\item \Ggfs sind auch personelle Ressourcen einzuplanen (\zB unterstützende Mitarbeiter).
	\item Hinweis: Häufig werden hier Ressourcen vergessen, die als selbstverständlich angesehen werden (\zB PC, Büro). 
\end{itemize}


\subsection{Entwicklungsprozess}
\label{sec:Entwicklungsprozess}
\begin{itemize}
	\item Welcher Entwicklungsprozess wird bei der Bearbeitung des Projekts verfolgt (\zB Wasserfall, agiler Prozess)?
\end{itemize}
