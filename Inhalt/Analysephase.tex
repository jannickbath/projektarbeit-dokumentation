% !TEX root = ../Projektdokumentation.tex
\section{Analysephase} 
\label{sec:Analysephase}


\subsection{Ist-Analyse} 
\label{sec:IstAnalyse}

Der aktuelle Prozess zur Überprüfung der Echtheit und Gültigkeit von Tickets und Gutscheinen basiert auf manuellen Methoden. Dies führt zu einer hohen Fehleranfälligkeit und ist zudem äußerst ineffizient. Die Mitarbeiter sind gezwungen, viel Zeit in die Überprüfung der Tickets zu investieren, was ihre Effizienz erheblich mindert.

Die Wünsche der Mitarbeiter beinhalten eine schnellere und zuverlässigere Methode zur Überprüfung von Tickets und Gutscheinen. Es besteht ein klarer Bedarf an einer automatisierten Lösung, die menschliche Fehler minimiert und den Verifizierungsprozess beschleunigt. Die Mitarbeiter wollen ihre Zeit effektiver nutzen und sich auf wichtigere Aufgaben konzentrieren können.

Das Ziel des Projekts ist es, ein QR-Code-basiertes Einlösungsprogramm zu erstellen, das diese Probleme adressiert. Durch die Implementierung eines Systems, das QR-Codes generiert und scannt, wird die Echtheit und Gültigkeit der Tickets schnell und sicher überprüft. Dies verbessert nicht nur die Effizienz der Arbeitsabläufe, sondern erhöht auch die Zuverlässigkeit und Sicherheit des gesamten Prozesses. Insgesamt führt die Automatisierung zu einer gesteigerten Zufriedenheit sowohl der Mitarbeiter als auch der Kunden.

\subsection{Wirtschaftlichkeitsanalyse}
\label{sec:Wirtschaftlichkeitsanalyse}
\begin{itemize}
	\item Lohnt sich das Projekt für das Unternehmen?
\end{itemize}


\subsubsection{\gqq{Make or Buy}-Entscheidung}
\label{sec:MakeOrBuyEntscheidung}

Bei der Entscheidung, ob eine bestehende Lösung verwendet oder eine neue entwickelt werden sollte, wurde sorgfältig geprüft, ob es bereits ein fertiges Produkt gibt, das alle Anforderungen des Projekts abdeckt. Dabei stellte sich heraus, dass es keine existierende Lösung gab, die den spezifischen Anforderungen entsprach.

Ein wichtiges Kriterium war die Kompatibilität mit dem bereits genutzten Framework Contao und dem Onlineshop-Plugin Isotope. Die Suche nach einer fertigen Lösung, die als Bundle über Composer installiert werden kann und nahtlos mit Contao und Isotope zusammenarbeitet, war erfolglos. Die vorhandenen Produkte auf dem Markt deckten entweder nicht alle funktionalen Anforderungen ab oder waren nicht vollständig kompatibel mit der bestehenden Architektur.

Aufgrund dieser speziellen Anforderungen und der fehlenden passenden Lösungen auf dem Markt wurde entschieden, das Projekt intern umzusetzen. Dies ermöglichte eine maßgeschneiderte Entwicklung, die exakt auf die Bedürfnisse des Unternehmens und der Mitarbeiter zugeschnitten ist. Durch die interne Entwicklung konnte sichergestellt werden, dass alle Anforderungen vollständig erfüllt und gleichzeitig die bestehende Architektur optimal genutzt wird.

\subsubsection{Projektkosten}
\label{sec:Projektkosten}
\begin{itemize}
	\item Welche Kosten fallen bei der Umsetzung des Projekts im Detail an (\zB Entwicklung, Einführung/Schulung, Wartung)?
\end{itemize}

\paragraph{Beispielrechnung (verkürzt)}
Die Kosten für die Durchführung des Projekts setzen sich sowohl aus Personal-, als auch aus Ressourcenkosten zusammen.
Laut Tarifvertrag verdient ein Auszubildender im dritten Lehrjahr pro Monat \eur{1000} Brutto. 

\begin{eqnarray}
8 \mbox{ h/Tag} \cdot 220 \mbox{ Tage/Jahr} = 1760 \mbox{ h/Jahr}\\
\eur{1000}\mbox{/Monat} \cdot 13,3 \mbox{ Monate/Jahr} = \eur{13300} \mbox{/Jahr}\\
\frac{\eur{13300} \mbox{/Jahr}}{1760 \mbox{ h/Jahr}} \approx \eur{7,56}\mbox{/h}
\end{eqnarray}

Es ergibt sich also ein Stundenlohn von \eur{7,56}. 
Die Durchführungszeit des Projekts beträgt 70 Stunden. Für die Nutzung von Ressourcen\footnote{Räumlichkeiten, Arbeitsplatzrechner etc.} wird 
ein pauschaler Stundensatz von \eur{15} angenommen. Für die anderen Mitarbeiter wird pauschal ein Stundenlohn von \eur{25} angenommen. 
Eine Aufstellung der Kosten befindet sich in Tabelle~\ref{tab:Kostenaufstellung} und sie betragen insgesamt \eur{2739,20}.
\tabelle{Kostenaufstellung}{tab:Kostenaufstellung}{Kostenaufstellung.tex}


\subsubsection{Amortisationsdauer}
\label{sec:Amortisationsdauer}
\begin{itemize}
	\item Welche monetären Vorteile bietet das Projekt (\zB Einsparung von Lizenzkosten, Arbeitszeitersparnis, bessere Usability, Korrektheit)?
	\item Wann hat sich das Projekt amortisiert?
\end{itemize}

\paragraph{Beispielrechnung (verkürzt)}
Bei einer Zeiteinsparung von 10 Minuten am Tag für jeden der 25 Anwender und 220 Arbeitstagen im Jahr ergibt sich eine gesamte Zeiteinsparung von 
\begin{eqnarray}
25 \cdot 220 \mbox{ Tage/Jahr} \cdot 10 \mbox{ min/Tag} = 55000 \mbox{ min/Jahr} \approx 917 \mbox{ h/Jahr} 
\end{eqnarray}

Dadurch ergibt sich eine jährliche Einsparung von 
\begin{eqnarray}
917 \mbox{h} \cdot \eur{(25 + 15)}{\mbox{/h}} = \eur{36680}
\end{eqnarray}

Die Amortisationszeit beträgt also $\frac{\eur{2739,20}}{\eur{36680}\mbox{/Jahr}} \approx 0,07 \mbox{ Jahre} \approx 4 \mbox{ Wochen}$.


\subsection{Nutzwertanalyse}
\label{sec:Nutzwertanalyse}
\begin{itemize}
	\item Darstellung des nicht-monetären Nutzens (\zB Vorher-/Nachher-Vergleich anhand eines Wirtschaftlichkeitskoeffizienten). 
\end{itemize}

\paragraph{Beispiel}
Ein Beispiel für eine Entscheidungsmatrix findet sich in Kapitel~\ref{sec:Architekturdesign}: \nameref{sec:Architekturdesign}.


\subsection{Anwendungsfälle}
\label{sec:Anwendungsfaelle}
\begin{itemize}
	\item Welche Anwendungsfälle soll das Projekt abdecken?
	\item Einer oder mehrere interessante (!) Anwendungsfälle könnten exemplarisch durch ein Aktivitätsdiagramm oder eine \ac{EPK} detailliert beschrieben werden. 
\end{itemize}

\paragraph{Beispiel}
Ein Beispiel für ein Use Case-Diagramm findet sich im \Anhang{app:UseCase}.


\subsection{Qualitätsanforderungen}
\label{sec:Qualitaetsanforderungen}
\begin{itemize}
	\item Welche Qualitätsanforderungen werden an die Anwendung gestellt (\zB hinsichtlich Performance, Usability, Effizienz \etc (siehe \citet{ISO9126}))?
\end{itemize}


\subsection{Lastenheft/Fachkonzept}
\label{sec:Lastenheft}
\begin{itemize}
	\item Auszüge aus dem Lastenheft/Fachkonzept, wenn es im Rahmen des Projekts erstellt wurde.
	\item Mögliche Inhalte: Funktionen des Programms (Muss/Soll/Wunsch), User Stories, Benutzerrollen
\end{itemize}

\paragraph{Beispiel}
Ein Beispiel für ein Lastenheft findet sich im \Anhang{app:Lastenheft}. 
