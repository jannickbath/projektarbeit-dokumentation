% !TEX root = ../Projektdokumentation.tex
\section{Analysephase} 
\label{sec:Analysephase}


\subsection{Ist-Analyse} 
\label{sec:IstAnalyse}

Der aktuelle Prozess zur Überprüfung der Echtheit und Gültigkeit von Tickets und Gutscheinen basiert auf manuellen Methoden. Dies führt zu einer hohen Fehleranfälligkeit und ist zudem äußerst ineffizient. Die Mitarbeiter sind gezwungen, viel Zeit in die Überprüfung der Tickets zu investieren, was ihre Effizienz erheblich mindert.

Die Wünsche der Mitarbeiter beinhalten eine schnellere und zuverlässigere Methode zur Überprüfung von Tickets und Gutscheinen. Es besteht ein klarer Bedarf an einer automatisierten Lösung, die menschliche Fehler minimiert und den Verifizierungsprozess beschleunigt. Die Mitarbeiter wollen ihre Zeit effektiver nutzen und sich auf wichtigere Aufgaben konzentrieren können.

Das Ziel des Projekts ist es, ein QR-Code-basiertes Einlösungsprogramm zu erstellen, das diese Probleme adressiert. Durch die Implementierung eines Systems, das QR-Codes generiert und scannt, wird die Echtheit und Gültigkeit der Tickets schnell und sicher überprüft. Dies verbessert nicht nur die Effizienz der Arbeitsabläufe, sondern erhöht auch die Zuverlässigkeit und Sicherheit des gesamten Prozesses. Insgesamt führt die Automatisierung zu einer gesteigerten Zufriedenheit sowohl der Mitarbeiter als auch der Kunden.

\subsection{Wirtschaftlichkeitsanalyse}
\label{sec:Wirtschaftlichkeitsanalyse}
\begin{itemize}
	\item Lohnt sich das Projekt für das Unternehmen?
\end{itemize}


\subsubsection{\gqq{Make or Buy}-Entscheidung}
\label{sec:MakeOrBuyEntscheidung}

Bei der Entscheidung, ob eine bestehende Lösung verwendet oder eine neue entwickelt werden sollte, wurde sorgfältig geprüft, ob es bereits ein fertiges Produkt gibt, das alle Anforderungen des Projekts abdeckt. Dabei stellte sich heraus, dass es keine existierende Lösung gab, die den spezifischen Anforderungen entsprach.

Ein wichtiges Kriterium war die Kompatibilität mit dem bereits genutzten Framework Contao und dem Onlineshop-Plugin Isotope. Die Suche nach einer fertigen Lösung, die als Bundle über Composer installiert werden kann und nahtlos mit Contao und Isotope zusammenarbeitet, war erfolglos. Die vorhandenen Produkte auf dem Markt deckten entweder nicht alle funktionalen Anforderungen ab oder waren nicht vollständig kompatibel mit der bestehenden Architektur.

Aufgrund dieser speziellen Anforderungen und der fehlenden passenden Lösungen auf dem Markt wurde entschieden, das Projekt intern umzusetzen. Dies ermöglichte eine maßgeschneiderte Entwicklung, die exakt auf die Bedürfnisse des Unternehmens und der Mitarbeiter zugeschnitten ist. Durch die interne Entwicklung konnte sichergestellt werden, dass alle Anforderungen vollständig erfüllt und gleichzeitig die bestehende Architektur optimal genutzt wird.

\subsubsection{Projektkosten}
\label{sec:Projektkosten}

\paragraph{Rechnung}
Die Kosten für die Durchführung des Projekts setzen sich sowohl aus Personal-, als auch aus Ressourcenkosten zusammen.
Laut Tarifvertrag verdient ein Auszubildender im dritten Lehrjahr pro Monat \eur{1000} Brutto. 

\begin{eqnarray}
8 \mbox{ h/Tag} \cdot 220 \mbox{ Tage/Jahr} = 1760 \mbox{ h/Jahr}\\
\eur{1000}\mbox{/Monat} \cdot 13,3 \mbox{ Monate/Jahr} = \eur{13300} \mbox{/Jahr}\\
\frac{\eur{13300} \mbox{/Jahr}}{1760 \mbox{ h/Jahr}} \approx \eur{7,56}\mbox{/h}
\end{eqnarray}

Es ergibt sich also ein Stundenlohn von \eur{7,56}. 
Die Durchführungszeit des Projekts beträgt 80 Stunden. Für die Nutzung von Ressourcen\footnote{Räumlichkeiten, Arbeitsplatzrechner etc.} wird 
ein pauschaler Stundensatz von \eur{15} angenommen. Für die anderen Mitarbeiter wird pauschal ein Stundenlohn von \eur{25} angenommen. 
Eine Aufstellung der Kosten befindet sich in Tabelle~\ref{tab:Kostenaufstellung} und sie betragen insgesamt \eur{2739,20}.
\tabelle{Kostenaufstellung}{tab:Kostenaufstellung}{Kostenaufstellung.tex}


\subsubsection{Amortisationsdauer}
\label{sec:Amortisationsdauer}

Da das System intern entwickelt wurde, entfallen Ausgaben für externe Softwarelizenzen, Updates und Wartungsverträge. Dies reduziert die langfristigen Betriebskosten erheblich und führt zu direkten Kosteneinsparungen.

Durch die interne Entwicklung des Systems vermeiden wir zudem die Kosten für die Anpassung und Integration externer Lösungen. Die Anwendung ist exakt auf die Bedürfnisse unserer Kunden und die von uns unterstützten Plattformen abgestimmt, was zusätzliche Anpassungsarbeiten überflüssig macht.

Ein weiterer signifikanter Vorteil ist die Zeitersparnis bei der Implementierung und dem Support. Dank der engen Integration mit dem bestehenden Contao-Framework und dem Isotope-Plugin können unsere Entwickler das System schneller und mit weniger Komplikationen einrichten. Die reduzierte Implementierungszeit bedeutet, dass wir mehr Projekte in kürzerer Zeit abschließen können, was zu einer höheren Umsatzrate führt.

Zusätzlich ermöglicht uns das QR-Code-System, unsere Marktposition zu stärken und als technologischer Vorreiter wahrgenommen zu werden. Die Fähigkeit, moderne Lösungen anzubieten, verbessert unsere Wettbewerbsfähigkeit und hilft dabei, neue Kunden zu gewinnen und bestehende Kundenbeziehungen zu vertiefen.

Durch den Verkauf des QR-Code-basierten Einlösungsprogramms können wir außerdem von wiederkehrenden Einnahmen profitieren. Wartungsverträge, Schulungen und zusätzliche Supportdienste bieten kontinuierliche Einkommensströme, die die finanzielle Stabilität unseres Unternehmens unterstützen.

\paragraph{Beispielrechnung (verkürzt)}
Bei einer Zeiteinsparung von 10 Minuten am Tag für jeden der 25 Anwender und 220 Arbeitstagen im Jahr ergibt sich eine gesamte Zeiteinsparung von 
\begin{eqnarray}
25 \cdot 220 \mbox{ Tage/Jahr} \cdot 10 \mbox{ min/Tag} = 55000 \mbox{ min/Jahr} \approx 917 \mbox{ h/Jahr} 
\end{eqnarray}

Dadurch ergibt sich eine jährliche Einsparung von 
\begin{eqnarray}
917 \mbox{h} \cdot \eur{(25 + 15)}{\mbox{/h}} = \eur{36680}
\end{eqnarray}

Die Amortisationszeit beträgt also $\frac{\eur{2739,20}}{\eur{36680}\mbox{/Jahr}} \approx 0,07 \mbox{ Jahre} \approx 4 \mbox{ Wochen}$.

\paragraph{Beispiel}
Ein Beispiel für eine Entscheidungsmatrix findet sich in Kapitel~\ref{sec:Architekturdesign}: \nameref{sec:Architekturdesign}.


\subsection{Anwendungsfälle}
\label{sec:Anwendungsfaelle}

Der erste Anwendungsfall betrifft die Generierung eines QR-Codes für eine Bestellung. Sobald ein Kunde im Onlineshop eine Bestellung abschließt, wird automatisch ein einzigartiger QR-Code generiert. Dieser QR-Code enthält alle relevanten Bestelldaten und wird in der Datenbank gespeichert. Anschließend wird der QR-Code per E-Mail an den Kunden versendet. Dies ermöglicht eine einfache und sichere Verifizierung der Bestellung.

Ein weiterer zentraler Anwendungsfall ist die Verifizierung und Einlösung des QR-Codes durch einen Mitarbeiter. Der Kunde präsentiert den QR-Code, den er per E-Mail erhalten hat. Der Mitarbeiter scannt den QR-Code, und das System überprüft dessen Gültigkeit. Ist der QR-Code gültig, werden dem Mitarbeiter alle relevanten Informationen zur Bestellung angezeigt. Der Mitarbeiter kann dann entscheiden, ob er den QR-Code tatsächlich einlösen möchte. Nach der Bestätigung wird der QR-Code im System als eingelöst markiert und kann nicht erneut verwendet werden. Dies stellt sicher, dass jeder QR-Code nur einmal genutzt werden kann und verhindert Betrugsversuche.

Zusätzlich gibt es die Möglichkeit einer manuelle Suche nach den Tickets. Sollte es einmal nicht möglich sein, den QR-Code zu scannen, kann der Mitarbeiter nach einem Ticket suchen. Das Programm zeigt dann die Gültigkeit des Tickets an und gibt die entsprechenden Bestellinformationen aus, um die Einlösung zu ermöglichen.

\paragraph{Use Case-Diagramm}
Ein Use Case-Diagramm befindet sich im \Anhang{app:UseCase}.

\subsection{Qualitätsanforderungen}
\label{sec:Qualitaetsanforderungen}

Ein zentraler Qualitätsaspekt ist die Fehlerfreiheit der Anwendung. Es sollten möglichst wenig Fehler in der Benutzung auftreten. Falls dennoch Fehler auftreten, muss klar und verständlich kommuniziert werden, was falsch gelaufen ist. Die Fehlermeldungen sollten in einer Form präsentiert werden, die dem Benutzer nützliche Informationen liefert, um das Problem zu beheben oder den Support zu kontaktieren. Kritische Fehler, die das gesamte Programm lahmlegen, sind unbedingt zu vermeiden.

Die Zuverlässigkeit des Systems ist ein weiterer wichtiger Faktor. Benutzer müssen sich darauf verlassen können, dass ein als gültig angezeigtes Ticket tatsächlich gültig ist. Das System sollte sich so gut wie möglich vor gefälschten oder manipulierten Tickets schützen. Ein Fehler in der Verifizierung der Tickets kann zu erheblichen finanziellen Verlusten für den Kunden führen. Daher muss das System stets korrekte und verlässliche Ergebnisse liefern.

Die Usability der Anwendung spielt ebenfalls eine große Rolle. Die Benutzeroberfläche sollte intuitiv und benutzerfreundlich gestaltet sein, sodass auch unerfahrene Benutzer problemlos mit dem System arbeiten können.

Die Integrationsfähigkeit in die bestehende Umgebung spielt auch eine entscheidende Rolle. Das Programm muss sich nahtlos in die vorhandene Infrastruktur integrieren lassen und die vorgegebene Struktur von Symfony Bundles einhalten. Dies gewährleistet eine reibungslose Integration in bestehende Systeme und erleichtert die Wartung und Erweiterung der Anwendung.

Das Programm muss außerdem sicherstellen, dass nur Mitarbeiter mit einem hinterlegten Login Zugang zu dem Einlösetool haben. Diese Maßnahme verhindert, dass Tickets versehentlich entwertet werden oder sensible Bestellinformationen von unbefugten Personen eingesehen werden können.

\subsection{Lastenheft/Fachkonzept}
\label{sec:Lastenheft}
\begin{itemize}
	\item Auszüge aus dem Lastenheft/Fachkonzept, wenn es im Rahmen des Projekts erstellt wurde.
	\item Mögliche Inhalte: Funktionen des Programms (Muss/Soll/Wunsch), User Stories, Benutzerrollen
\end{itemize}

\paragraph{Beispiel}
Ein Beispiel für ein Lastenheft findet sich im \Anhang{app:Lastenheft}. 
