% !TEX root = ../Projektdokumentation.tex
\section{Dokumentation}
\label{sec:Dokumentation}

Für die Benutzer der Anwendung wurde eine Benutzerdokumentation erstellt. Diese Dokumentation enthält detaillierte Anweisungen zur Nutzung der verschiedenen Funktionen der Anwendung, wie das Scannen und Einlösen von QR-Codes, sowie Tipps zur Fehlerbehebung und zum Umgang mit möglichen Problemen. Ziel war es, die Benutzer schrittweise durch die Anwendung zu führen und ihnen zu ermöglichen, das Programm auch ohne großen Schulungsaufwand zu verstehen.

Ein Teil der Benutzerdokumentation ist direkt im Programm implementiert, sodass Benutzer schnell darauf zugreifen können.

Für die Entwickler wurde der Quellcode mit PHPDoc-Kommentaren versehen. Diese Kommentare bieten eine klare und präzise Beschreibung der Funktionen, Klassen und Methoden im Code. PHPDoc erleichtert es den Entwicklern, den Code zu verstehen, zu warten und bei Bedarf zu erweitern.

Es wurde keine separate Entwicklerdokumentation erstellt. Stattdessen wurde darauf geachtet, dass der Code selbst gut strukturiert und kommentiert ist.


Ein Beispiel der dokumentierten PHP-Funktionen befindet sich im \Anhang{app:Doc}.
