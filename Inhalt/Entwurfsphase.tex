% !TEX root = ../Projektdokumentation.tex
\section{Entwurfsphase} 
\label{sec:Entwurfsphase}

\subsection{Zielplattform}
\label{sec:Zielplattform}

\textbf{Programmiersprache:} Die Anwendung wird in PHP unter Verwendung des Symfony-Frameworks entwickelt. Symfony bietet eine robuste und flexible Grundlage für Webanwendungen und unterstützt die Erstellung von sicheren und skalierbaren Lösungen. Die Wahl von PHP und Symfony ermöglicht eine effiziente Entwicklung und Wartung der Anwendung.

\textbf{Datenbank:} Als Datenbank wird MySQL eingesetzt. MySQL ist eine weit verbreitete, leistungsstarke und zuverlässige relationale Datenbank, die gut mit Symfony und PHP integriert werden kann. Sie bietet die notwendige Leistung und Stabilität, um die Bestelldaten und QR-Codes sicher zu speichern und schnell abzurufen.

\textbf{Client/Server-Architektur:} Die Anwendung wird als Webanwendung mit einer Client/Server-Architektur implementiert. Dies ermöglicht eine zentrale Verwaltung und Aktualisierung der Anwendung auf dem Server, während die Benutzer über ihre Webbrowser auf die Anwendung zugreifen. Die Verwendung einer Webanwendung stellt sicher, dass keine zusätzliche Software auf den Endgeräten der Benutzer installiert werden muss.

\textbf{Hardware:} Die Zielplattform umfasst Smartphones, da diese über integrierte Kameras verfügen, die für das Scannen der QR-Codes erforderlich sind. Die Anwendung soll auf allen Smartphones mit Webbrowsern und einer Internetverbindung zugänglich sein. Dies stellt sicher, dass alle Mitarbeiter unabhängig vom verwendeten Gerät auf die Anwendung zugreifen können.

Die Entscheidung für PHP, Symfony und MySQL fiel, weil diese Technologien in unserem Betrieb gängig sind und von unseren Entwicklern routiniert verwendet werden. Dadurch kann effizient und schnell gearbeitet werden. Zudem ermöglicht diese Kombination die Erstellung von Composer-Bundles, die sich nahtlos in andere Symfony-Projekte integrieren lassen.

Besonders wichtig ist dies, weil das Kundenprojekt, in dem das Bundle eingebunden werden soll, Contao verwendet, welches auf Symfony basiert. Diese Auswahl an Technologien erleichtert die Implementierung und gewährleistet eine konsistente Benutzererfahrung.

\subsection{Architekturdesign}
\label{sec:Architekturdesign}
\begin{itemize}
	\item Beschreibung und Begründung der gewählten Anwendungsarchitektur (\zB \acs{MVC}).
	\item \Ggfs Bewertung und Auswahl von verwendeten Frameworks sowie \ggfs eine kurze Einführung in die Funktionsweise des verwendeten Frameworks.
\end{itemize}

\paragraph{Beispiel}
Anhand der Entscheidungsmatrix in Tabelle~\ref{tab:Entscheidungsmatrix} wurde für die Implementierung der Anwendung das \acs{PHP}-Framework Symfony\footnote{\Vgl \citet{Symfony}.} ausgewählt. 

\tabelle{Entscheidungsmatrix}{tab:Entscheidungsmatrix}{Nutzwert.tex}


\subsection{Entwurf der Benutzeroberfläche}
\label{sec:Benutzeroberflaeche} 
\begin{itemize}
	\item Entscheidung für die gewählte Benutzeroberfläche (\zB GUI, Webinterface).
	\item Beschreibung des visuellen Entwurfs der konkreten Oberfläche (\zB Mockups, Menüführung).
	\item \Ggfs Erläuterung von angewendeten Richtlinien zur Usability und Verweis auf Corporate Design.
\end{itemize}

\paragraph{Beispiel}
Beispielentwürfe finden sich im \Anhang{app:Entwuerfe}.


\subsection{Datenmodell}
\label{sec:Datenmodell}

\begin{itemize}
	\item Entwurf/Beschreibung der Datenstrukturen (\zB \acs{ERM} und/oder Tabellenmodell, \acs{XML}-Schemas) mit kurzer Beschreibung der wichtigsten (!) verwendeten Entitäten.
\end{itemize}

\paragraph{Beispiel}
In \Abbildung{ER} wird ein \ac{ERM} dargestellt, welches lediglich Entitäten, Relationen und die dazugehörigen Kardinalitäten enthält. 

\begin{figure}[htb]
\centering
\includegraphicsKeepAspectRatio{ERDiagramm.pdf}{0.6}
\caption{Vereinfachtes ER-Modell}
\label{fig:ER}
\end{figure} 


\subsection{Geschäftslogik}
\label{sec:Geschaeftslogik}

\begin{itemize}
	\item Modellierung und Beschreibung der wichtigsten (!) Bereiche der Geschäftslogik (\zB mit Kom\-po\-nen\-ten-, Klassen-, Sequenz-, Datenflussdiagramm, Programmablaufplan, Struktogramm, \ac{EPK}).
	\item Wie wird die erstellte Anwendung in den Arbeitsfluss des Unternehmens integriert?
\end{itemize}

\paragraph{Beispiel}
Ein Klassendiagramm, welches die Klassen der Anwendung und deren Beziehungen untereinander darstellt kann im \Anhang{app:Klassendiagramm} eingesehen werden.

\Abbildung{Modulimport} zeigt den grundsätzlichen Programmablauf beim Einlesen eines Moduls als \ac{EPK}.
\begin{figure}[htb]
\centering
\includegraphicsKeepAspectRatio{modulimport.pdf}{0.9}
\caption{Prozess des Einlesens eines Moduls}
\label{fig:Modulimport}
\end{figure}


\subsection{Maßnahmen zur Qualitätssicherung}
\label{sec:Qualitaetssicherung}
\begin{itemize}
	\item Welche Maßnahmen werden ergriffen, um die Qualität des Projektergebnisses (siehe Kapitel~\ref{sec:Qualitaetsanforderungen}: \nameref{sec:Qualitaetsanforderungen}) zu sichern (\zB automatische Tests, Anwendertests)?
	\item \Ggfs Definition von Testfällen und deren Durchführung (durch Programme/Benutzer).
\end{itemize}


\subsection{Pflichtenheft/Datenverarbeitungskonzept}
\label{sec:Pflichtenheft}
\begin{itemize}
	\item Auszüge aus dem Pflichtenheft/Datenverarbeitungskonzept, wenn es im Rahmen des Projekts erstellt wurde.
\end{itemize}

\paragraph{Beispiel}
Ein Beispiel für das auf dem Lastenheft (siehe Kapitel~\ref{sec:Lastenheft}: \nameref{sec:Lastenheft}) aufbauende Pflichtenheft ist im \Anhang{app:Pflichtenheft} zu finden.
