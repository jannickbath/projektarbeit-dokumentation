% !TEX root = ../Projektdokumentation.tex
\section{Implementierungsphase} 
\label{sec:Implementierungsphase}

\subsection{Implementierung der Datenstrukturen}
\label{sec:ImplementierungDatenstrukturen}

Contao generiert die Datenbanktabellen automatisch anhand der DCA-Dateien, die im \texttt{dca/} Ordner des jeweiligen Bundles hinterlegt sind. Eine DCA-Datei enthält die Konfiguration für eine Datenbanktabelle, einschließlich der Definition der Felder und deren Eigenschaften. Beim Installations- oder Aktualisierungsprozess liest Contao diese Konfigurationsdateien aus und erstellt oder aktualisiert die entsprechenden Tabellen in der Datenbank.

Die DCA-Datei definiert die Struktur und die Felder der Tabelle, einschließlich der SQL-Typen und anderer relevanter Einstellungen. Contao interpretiert diese Informationen und generiert die erforderlichen SQL-Befehle, um die Datenbanktabellen anzulegen oder zu aktualisieren.

Die Tabelle \texttt{tl\_iso\_order\_hashes} wurde erstellt, um die QR-Codes und deren zugehörige Informationen zu verwalten.

\begin{itemize}
    \item \textbf{id}: Dieses Feld dient als Primärschlüssel und wird automatisch hochgezählt. Es wird als \texttt{int(10) unsigned} definiert, um eine ausreichende Anzahl von Einträgen zu ermöglichen.
    \item \textbf{hash}: Dieses Feld speichert den aus Bestellinformationen generierten Hash-Wert. Es ist als \texttt{varchar(64)} definiert, um den Hash-Wert aufzunehmen, wobei 64 Zeichen ausreichend für einen sicheren Hash-Wert sind.
    \item \textbf{iso\_product\_collection\_id}: Dieses Feld speichert die ID der zugehörigen Produktkollektion aus dem Isotope-Shop. Es wird als \texttt{int(10)} definiert.
    \item \textbf{valid}: Dieses Feld definiert die Gültigkeit des QR-Codes. Es ist als \texttt{int(1)} definiert, wobei der Standardwert \texttt{0} ist. Ein Wert von \texttt{1} bedeutet, dass der QR-Code gültig und noch nicht eingelöst ist.
\end{itemize}

Ein Screenshot der DCA-Datei wird im Anhang bereitgestellt, um eine visuelle Darstellung der Konfiguration zu bieten. Der Screenshot ist hier zu finden: \Anhang{app:ScreenshotDca}.

\subsection{Implementierung der Benutzeroberfläche}
\label{sec:ImplementierungBenutzeroberflaeche}

Die Implementierung der Benutzeroberfläche für das QR-Code-basierte Einlösungsprogramm erfolgte unter Verwendung von HTML5-Templates, die sowohl HTML als auch PHP beinhalten. Diese Kombination ermöglicht eine flexible und dynamische Darstellung der Inhalte, die durch die Backend-Logik gesteuert wird.

Für das Styling der Benutzeroberfläche kamen SCSS-Dateien zum Einsatz. SCSS, als Erweiterung von CSS, bietet zusätzliche Funktionen wie Variablen, geschachtelte Regeln und Mixins, die die Erstellung und Verwaltung der Stylesheets erheblich vereinfachen. Diese SCSS-Dateien wurden mithilfe des SASS Pre-Processors in reguläres CSS kompiliert und anschließend in die HTML5-Templates eingebunden.

Die Logik im Frontend wurde mit JavaScript realisiert. JavaScript wurde verwendet, um interaktive Elemente in die Benutzeroberfläche zu integrieren. Dies umfasst unter anderem die Implementierung der QR-Code-Scanfunktion, die Überprüfung und Validierung der Eingaben sowie die Anzeige von Bestellinformationen und Fehlermeldungen.

Für das Scannen der QR-Codes wurde die externe Bibliothek \texttt{html5-qrcode} verwendet. Diese Bibliothek bietet eine leistungsfähige und benutzerfreundliche Möglichkeit, QR-Codes direkt im Browser mithilfe der Gerätekamera zu scannen.

Die Gestaltung der Benutzeroberfläche basiert auf den vorgegebenen Figma-Designs. Diese Designs wurden als Grundlage für die Entwicklung verwendet, um sicherzustellen, dass die Benutzeroberfläche sowohl ästhetisch ansprechend als auch funktional ist.

Screenshots der Anwendung befinden sich im \Anhang{Screenshots}.


\subsection{Implementierung der Geschäftslogik}
\label{sec:ImplementierungGeschaeftslogik}

\begin{itemize}
	\item Beschreibung des Vorgehens bei der Umsetzung/Programmierung der entworfenen Anwendung.
	\item \Ggfs interessante Funktionen/Algorithmen im Detail vorstellen, verwendete Entwurfsmuster zeigen.
	\item Quelltextbeispiele zeigen.
	\item Hinweis: Wie in Kapitel~\ref{sec:Einleitung}: \nameref{sec:Einleitung} zitiert, wird nicht ein lauffähiges Programm bewertet, sondern die Projektdurchführung. Dennoch würde ich immer Quelltextausschnitte zeigen, da sonst Zweifel an der tatsächlichen Leistung des Prüflings aufkommen können.
\end{itemize}

\paragraph{Beispiel}
Die Klasse \texttt{Com\-par\-ed\-Na\-tu\-ral\-Mo\-dule\-In\-for\-ma\-tion} findet sich im \Anhang{app:CNMI}.  
