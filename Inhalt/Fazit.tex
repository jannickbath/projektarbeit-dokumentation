% !TEX root = ../Projektdokumentation.tex
\section{Fazit} 
\label{sec:Fazit}

\subsection{Soll-/Ist-Vergleich}
\label{sec:SollIstVergleich}

Das Projektziel, ein QR-Code-basiertes Einlösungsprogramm zu entwickeln, wurde in vollem Umfang erreicht. Alle funktionalen Anforderungen wurden erfolgreich umgesetzt. Das System generiert zuverlässig eindeutige QR-Codes für jede Bestellung, ermöglicht eine einfache und sichere Einlösung der Tickets und bietet eine benutzerfreundliche Oberfläche.

Der Auftraggeber ist mit dem Projektergebnis äußerst zufrieden. Die Anwendung wurde erfolgreich in Betrieb genommen und erfüllt alle gestellten Anforderungen und Erwartungen.

In Bezug auf die Projektplanung gab es nur minimale Abweichungen. Die Dokumentation des Projekts nahm etwas mehr Zeit in Anspruch als ursprünglich geplant. Dieser zusätzliche Zeitaufwand war notwendig, um sicherzustellen, dass alle Aspekte des Projekts umfassend und verständlich dokumentiert wurden.

Insgesamt kann das Projekt als erfolgreich abgeschlossen betrachtet werden. Die entwickelten Lösungen entsprechen den Anforderungen und Erwartungen des Auftraggebers, und die wenigen Abweichungen in der Zeitplanung haben das Gesamtergebnis nicht negativ beeinflusst.

Wie in Tabelle~\ref{tab:Vergleich} zu erkennen ist, konnte die Zeitplanung bis auf wenige Ausnahmen eingehalten werden.
\tabelle{Soll-/Ist-Vergleich}{tab:Vergleich}{Zeitnachher.tex}


\subsection{Lessons Learned}
\label{sec:LessonsLearned}

Es hat sich gezeigt, dass die Dokumentation eines Projekts sehr zeitintensiv ist und sorgfältige Planung erfordert, um alle Aspekte umfassend und präzise darzustellen. 

Diese Erfahrung hat den Prüfling dazu veranlasst, zukünftig mehr Zeit für die Dokumentationsphase einzuplanen und diese bereits in der frühen Planungsphase des Projekts stärker zu berücksichtigen.

\subsection{Ausblick}
\label{sec:Ausblick}

Es wird davon ausgegangen, dass das Projekt zukünftig in mehrere Online-Shops eingebunden wird. Die Flexibilität und Kompatibilität der Anwendung mit Isotope-Shops ermöglichen eine einfache Integration.

Dank der modularen Struktur und der Nutzung von Symfony-Standards kann das Programm relativ einfach an die spezifischen Bedürfnisse und Anforderungen unterschiedlicher Online-Shops angepasst werden.